\begin{appendices}
\chapter{Methodologies}
\section{Long Short-Term Memory and Gated Recurrent Unit} \label{app:LSTM_GRU}

\begin{center}
\begin{figure}[H]
    \raisebox{-0.5\height}{\includegraphics[width=0.45\textwidth]{"./images/LSTM".png}}
    \raisebox{-0.5\height}{\includegraphics[width=0.45\textwidth]{"./images/GRU".png}}
    \caption{A visual representation of the LSTM cell and the GRU cell}
    \label{app::LSTM_GRU}
\end{figure}
\end{center}

The LSTM and the GRU cell units are shown in Figure \ref{app::LSTM_GRU}, where $\sigma$ depicts the sigmoid function. The equations that govern the LSTM are
\begin{equation*}
\begin{cases}
\begin{pmatrix} i \\ f \\ o \\ g \end{pmatrix} = \begin{pmatrix} \sigma \\ \sigma \\ \sigma \\ \tanh \end{pmatrix} W \begin{pmatrix} h_{t-1} \\ x_t \end{pmatrix} \\
c_t f \cdot c_{t-1} + i \cdot g \\
h_t = o \cdot \tanh(c_t)
\end{cases}
\end{equation*}

The equations that govern the GRU are
\begin{equation*}
\begin{cases}
\text{Update Gate: } &  z_t = \sigma(W_z [h_{t-1},x_t]) \\
\text{Reset Gate: } & r_t = \sigma(W_r[h_{t-1},x_t]) \\
\text{Current Memory Content: } & \tilde{h}_t = \tanh(W \cdot [r_t \cdot h_{t-1}, x_t]) \\
\text{Final Memory: } & h_t = (1-z_t) \cdot h_{t-1} + z_t \cdot \tilde{h}_t
\end{cases}
\end{equation*}

Although sigmoid functions are often frowned upon in the deep learning community due to the inevitable fact that often times, the gradients become zero at very high or low values, the sigmoids behave as switch knobs for the acceptance or rejection of previous information on the cell memory $c_t$ or on the hidden states $h_t$, which are coming from previous iterations. The vanishing gradient problem is avoided in LSTM and GRU or any network with forget gates through the concept of linear carousel.
% \section{Gaussian Mixture Models} \label{app:gmm}


\section{Spectral Clustering} \label{app:spectralclustering}
% \section{Mel-Frequency Cepstrum Coefficients}
% \section{Linear Prediction Coefficients and its Derivatives}

\end{appendices}
